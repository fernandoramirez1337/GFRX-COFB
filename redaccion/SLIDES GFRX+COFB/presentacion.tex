\documentclass[aspectratio=169]{beamer}
\usetheme{Madrid}
\usecolortheme{default}

\usepackage[utf8]{inputenc}
\usepackage[spanish]{babel}
\usepackage{graphicx}
\usepackage{booktabs}
\usepackage{multirow}
\usepackage{array}
\usepackage{colortbl}
\usepackage[table]{xcolor}

\title{Implementación de Cifrado Autenticado con Datos Asociados GFRX+COFB para entornos IoT}
\author{Fernando Ramirez Arredondo\\
\vspace{0.3cm}
\small Asesor: Dr. Yván Jesús Túpac Valdivia\\
\vspace{0.3cm}
Universidad Católica San Pablo}
\date{Arequipa, Octubre 2025}

\begin{document}

% Slide 1: Título
\begin{frame}
\titlepage
\end{frame}

% Slide 2: Motivación - Crecimiento IoT
\begin{frame}{Crecimiento Exponencial del IoT}
\begin{figure}
\centering
\includegraphics[width=0.85\textwidth]{../figs/grafico_iot_statista_2024.png}
\end{figure}
\vspace{-0.3cm}
\centering
\small{41.1 mil millones de dispositivos conectados proyectados para 2030}
\end{frame}

% Slide 3: Problema
\begin{frame}{El Problema}
\begin{center}
\Large
\textbf{Dispositivos IoT con recursos limitados}
\vspace{1cm}

\normalsize
\begin{itemize}
\item RAM: $<$10 KB
\item Procesadores: 8-16 bits
\item LUTs disponibles: $<$5000
\end{itemize}

\vspace{1cm}
\Large
AES-GCM requiere $\sim$3175 LUTs
\end{center}
\end{frame}

% Slide 4: Objetivos
\begin{frame}{Objetivo General}
\begin{center}
\Large
Desarrollar la primera implementación funcional de GFRX+COFB
\vspace{1cm}

\normalsize
\begin{itemize}
\item Implementación en C para validación funcional
\item Síntesis HLS para estimación de características hardware
\item Estado mínimo de 320 bits (1.5n+k)
\end{itemize}
\end{center}
\end{frame}

% Slide 5: Parámetros GFRX
\begin{frame}{Parámetros de GFRX-128/128}
\begin{center}
\begin{tabular}{lc}
\toprule
\textbf{Parámetro} & \textbf{Valor} \\
\midrule
Tamaño de bloque & 128 bits \\
Tamaño de clave & 128 bits \\
Número de rondas & 32 \\
Operaciones por ronda & 3 AND/ADD, 4 ROT, 5 XOR \\
Estructura & Feistel generalizada (4 ramas) \\
\rowcolor{green!30}
\textbf{Difusión completa} & \textbf{6 rondas} \\
\bottomrule
\end{tabular}
\end{center}
\end{frame}

% Slide 6: Comparación Modos AEAD
\begin{frame}{Modos AEAD Ligeros - Comparación}
\begin{center}
\small
\begin{tabular}{lccc}
\toprule
\textbf{Modo} & \textbf{Estado (bits)} & \textbf{Tasa} & \textbf{Característica Principal} \\
\midrule
JAMBU & 1.5n & 1 bloque & Estado mínimo \\
\rowcolor{yellow!30}
\textbf{COFB} & \textbf{1.5n+k} & \textbf{1 bloque} & \textbf{Retroalimentación combinada} \\
OCB & 2n+k & 1 bloque & Paralelizable \\
ASCON & Variable & r bits & Hash integrado \\
\bottomrule
\end{tabular}
\end{center}
\vspace{0.3cm}
\centering
COFB: Diseño minimalista con máscaras de 64 bits en GF($2^{64}$)
\end{frame}

% Slide 7: Características COFB
\begin{frame}{Características del Modo COFB}
\begin{center}
\begin{tabular}{lc}
\toprule
\textbf{Característica} & \textbf{Valor} \\
\midrule
\rowcolor{green!30}
\textbf{Estado requerido} & \textbf{1.5n + k bits} \\
Tamaño de máscara & n/2 bits (64 bits) \\
Tasa de procesamiento & 1 bloque por invocación \\
Paralelizable & No \\
Campo finito & GF($2^{64}$) \\
Seguridad & $2^{64}$ consultas \\
\bottomrule
\end{tabular}
\end{center}
\vspace{0.3cm}
\centering
Para n=128, k=128: Estado total = \textbf{320 bits}
\end{frame}

% Slide 8: Arquitectura Propuesta
\begin{frame}{Arquitectura GFRX+COFB}
\begin{center}
\includegraphics[height=0.7\textheight]{../figs/arquitectura_gfrx_cofb.png}
\end{center}
\vspace{-0.2cm}
\centering
\tiny{5 módulos: GFRX\_Core, COFB\_Controller, Feedback\_Function, Mask\_Generator, Interface\_Module}
\end{frame}

% Slide 9: Metodología
\begin{frame}{Metodología de Diseño}
\begin{center}
\Large
\textbf{Enfoque en dos etapas}
\vspace{1cm}

\begin{columns}
\column{0.45\textwidth}
\centering
\textbf{Etapa 1}
\vspace{0.3cm}

Implementación en C
\begin{itemize}
\item Validación funcional
\item Métricas software
\item Profiling
\end{itemize}

\column{0.45\textwidth}
\centering
\textbf{Etapa 2}
\vspace{0.3cm}

Síntesis HLS
\begin{itemize}
\item LegUp / Bambu
\item Estimación hardware
\item Caracterización
\end{itemize}
\end{columns}
\end{center}
\end{frame}

% Slide 10: Plan de Implementación
\begin{frame}{Plan de Implementación}
\begin{center}
\small
\begin{tabular}{clc}
\toprule
\textbf{Fase} & \textbf{Actividad} & \textbf{Objetivo} \\
\midrule
1 & Implementación GFRX-128 en C & Núcleo funcional \\
2 & Implementación COFB en C & Modo AEAD completo \\
3 & Validación funcional exhaustiva & Correctitud verificada \\
4 & Síntesis HLS & Estimación hardware \\
\bottomrule
\end{tabular}
\end{center}
\vspace{0.5cm}
\centering
3 niveles de validación: Vectores unitarios, integración, compliance NIST
\end{frame}

% Slide 11: Comparación con Esquemas Establecidos
\begin{frame}{Comparación con Esquemas AEAD Establecidos}
\begin{center}
\tiny
\begin{tabular}{lccccc}
\toprule
\textbf{Esquema} & \textbf{Primitiva} & \textbf{Estado} & \textbf{LUTs} & \textbf{Throughput} & \textbf{Tipo} \\
 & & \textbf{(bits)} & \textbf{(aprox)} & \textbf{@100MHz} & \textbf{Impl.} \\
\midrule
AES-GCM & AES-128 & 384 & $\sim$3175 & 1.28 Gbps & FPGA Real \\
ASCON & Permutación & 320 & $\sim$1712 & 640 Mbps & FPGA Real \\
GIFT-COFB & GIFT-128 & 320 & $\sim$1450 & 400 Mbps & FPGA Real \\
TinyJAMBU & TJ-128 & 288 & $\sim$800 & 200 Mbps & FPGA Real \\
\midrule
\rowcolor{yellow!30}
\textbf{GFRX+COFB} & \textbf{GFRX-128} & \textbf{320} & \textbf{A medir} & \textbf{A medir} & \textbf{HLS Est.} \\
\bottomrule
\end{tabular}
\end{center}
\centering
\end{frame}

% Slide 12: Resultados Esperados
\begin{frame}{Resultados Esperados}
\begin{center}
\begin{itemize}
\item \textbf{Implementación funcional en C} completa y validada
\vspace{0.3cm}
\item \textbf{Métricas de software} mediante profiling (Valgrind, gprof)
\vspace{0.3cm}
\item \textbf{Estado mínimo verificado:} 320 bits totales
\vspace{0.3cm}
\item \textbf{Estimaciones HLS:} Área, throughput y consumo energético
\vspace{0.3cm}
\item \textbf{Seguridad teórica:} GFRX (40\% margen) + COFB ($2^{64}$ consultas)
\vspace{0.3cm}
\item \textbf{Comparación cualitativa} con AES-GCM, ASCON, GIFT-COFB
\end{itemize}
\end{center}
\end{frame}

% Slide 13: Conclusiones
\begin{frame}{Conclusiones}
\begin{center}
\Large
\vspace{1cm}

\normalsize
\begin{itemize}
\item Estado mínimo: 320 bits (uno de los más bajos)
\vspace{0.3cm}
\item Metodología: C + HLS para estimación hardware
\vspace{0.3cm}
\item Potencial alternativa competitiva para IoT
\vspace{0.3cm}
\item Trabajo futuro: Implementación RTL y validación en hardware real
\end{itemize}
\end{center}
\end{frame}

% Slide 14: Agradecimientos
\begin{frame}
\begin{center}
\Huge
¿Preguntas?
\vspace{2cm}

\normalsize
Gracias por su atención
\end{center}
\end{frame}

\end{document}
