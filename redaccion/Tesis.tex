% Template LaTeX Tesis Ciencia de la Computación UCSP
% 2022

\documentclass[a4paper,openany,12pt]{book}
\usepackage{amsmath}
\usepackage{amssymb}
\usepackage{float}
\usepackage[utf8]{inputenc}
\usepackage{graphicx}
\usepackage[spanish,mexico]{babel}
\usepackage{sty/fancyhdr}
\usepackage{ae}
\usepackage[left=2.5cm,right=2.5cm,top=3cm,bottom=2cm]{geometry}
\usepackage[printonlyused]{acronym}
\usepackage{xspace}
\usepackage{sty/hlundef}
\usepackage{sty/tesis}
\usepackage{setspace}
\usepackage{booktabs}
\usepackage{multirow}
\usepackage{rotating}
\usepackage{algorithm}
\usepackage{algorithmic}
\usepackage{listings}
\usepackage{hyperref}
\usepackage{url}
\usepackage{microtype}

% Configuración de listings para código
\lstset{
    basicstyle=\ttfamily\footnotesize,
    breaklines=true,
    columns=fullflexible,
    keepspaces=true,
    showstringspaces=false,
    numbers=left,
    numberstyle=\tiny,
    frame=single,
    captionpos=b,
    tabsize=4
}

\title{Implementación de Cifrado Autenticado con Datos Asociados GFRX+COFB para entornos IoT}


\author{Fernando Ramirez Arredondo}


\advisor{Dr. Yván Jesús Túpac Valdivia}


\date{Arequipa, Octubre 2025}


\dedicado{A mis padres, por todo lo que me han dado.}

\begin{document}
\pagestyle{fancy}

\maketitle %Compone la carátula y la dedicatoria
\newpage

%\approved{\tres}%  {\tres} or {\cuatro}

\chapter*{Abreviaturas}
\addcontentsline{toc}{chapter}{Abreviaturas}

\begin{acronym}[INT-CTXT]
\acro{AEAD}{Authenticated Encryption with Associated Data}
\acro{AES}{Advanced Encryption Standard}
\acro{ARX}{Addition, Rotation, XOR}
\acro{COFB}{COmbined FeedBack}
\acro{FPGA}{Field-Programmable Gate Array}
\acro{GCM}{Galois/Counter Mode}
\acro{GFRX}{Generalized Feistel with Rotation and XOR}
\acro{GIFT}{GIFT block cipher}
\acro{HDL}{Hardware Description Language}
\acro{HLS}{High-Level Synthesis}
\acro{IND-CPA}{Indistinguishability under Chosen Plaintext Attack}
\acro{INT-CTXT}{Integrity of Ciphertext}
\acro{IoT}{Internet of Things}
\acro{LUT}{Look-Up Table}
\acro{NIST}{National Institute of Standards and Technology}
\acro{RAM}{Random Access Memory}
\acro{RTL}{Register Transfer Level}
\acro{SPN}{Substitution-Permutation Network}
\acro{XOR}{Exclusive OR}
\end{acronym}

\newpage

\begin{center}
\textbf{\MakeUppercase{Agradecimientos}}
\end{center}

\vspace{1cm}

A mi familia, por su apoyo incondicional durante todo este proceso.

A mi asesor, Dr. Yván Jesús Túpac Valdivia, por su guía, paciencia y valiosos aportes a lo largo de esta investigación.

A la Universidad Católica San Pablo, por brindarme las herramientas y el espacio necesario para desarrollar este trabajo.

A todos quienes de una u otra forma contribuyeron a la realización de esta tesis.

\newpage
 %Inserta los agradecimientos
\begin{center}
\textbf{\MakeUppercase{Resumen}}
\end{center}

\vspace{1cm}

El Internet de las Cosas (IoT) ha transformado radicalmente la interacción entre dispositivos físicos y sistemas digitales, con más de 40 mil millones de dispositivos conectados proyectados para 2030. Sin embargo, estos dispositivos enfrentan desafíos críticos de seguridad debido a sus recursos computacionales extremadamente limitados. Los esquemas tradicionales de cifrado autenticado con datos asociados (AEAD), como AES-GCM, resultan computacionalmente costosos para sensores IoT con menos de 10KB de RAM y procesadores de 8-16 bits.

Esta tesis propone la primera implementación del cifrado autenticado GFRX+COFB, combinando el cifrado de bloque ligero GFRX con el modo de operación COFB. La implementación se desarrollará en lenguaje C para validación funcional y medición de rendimiento en software, utilizando posteriormente herramientas de síntesis de alto nivel para estimar características de hardware. GFRX, basado en una estructura Feistel generalizada con operaciones ARX, alcanza difusión completa en solo 6 rondas. El modo COFB minimiza el estado requerido a 1.5n+k bits (donde n es el tamaño de bloque y k el tamaño de clave), resultando en un esquema AEAD de solo 320 bits de estado total.

\textbf{Palabras clave:} Criptografía ligera, AEAD, GFRX, COFB, Internet de las Cosas, cifrado autenticado, dispositivos con recursos limitados.

\newpage
 %Inserta el resumen
\begin{center}
\textbf{\MakeUppercase{Abstract}}
\end{center}

\vspace{1cm}

The Internet of Things (IoT) has radically transformed the interaction between physical devices and digital systems, with over 40 billion connected devices projected by 2030. However, these devices face critical security challenges due to their extremely limited computational resources. Traditional Authenticated Encryption with Associated Data (AEAD) schemes, such as AES-GCM, are computationally expensive for IoT sensors with less than 10KB of RAM and 8-16 bit processors.

This thesis proposes the first implementation of GFRX+COFB authenticated encryption, combining the lightweight block cipher GFRX with the COFB operating mode. The implementation will be developed in C language for functional validation and software performance measurement, subsequently using high-level synthesis tools to estimate hardware characteristics. GFRX, based on a generalized Feistel structure with ARX operations, achieves full diffusion in only 6 rounds. The COFB mode minimizes the required state to 1.5n+k bits (where n is the block size and k the key size), resulting in an AEAD scheme with only 320 bits of total state.

\textbf{Keywords:} Lightweight cryptography, AEAD, GFRX, COFB, Internet of Things, authenticated encryption, resource-constrained devices.

\newpage
 %Inserta el abstract

\pagenumbering{roman}
\setcounter{page}{1}
\pagestyle{plain}

\tableofcontents %Inserta el índice general
\listoftables %Inserta el índice de cuadros
\listoffigures %Inserta el índice de figuras

%%%%%%%%%%%%%%%%%%%%%%%%%%%%%%%%%%%%%%%%%%%%%%%%%%%%%%%%%%%%%%%%%%%%%
%%%%   En esta parte deberas incluir los archivos de tu tesis   %%%%%
%%%%%%%%%%%%%%%%%%%%%%%%%%%%%%%%%%%%%%%%%%%%%%%%%%%%%%%%%%%%%%%%%%%%%

\pagestyle{plain}
\pagenumbering{arabic}
\setcounter{page}{1}
\chapter{Introducción}

Este documento presenta una propuesta de tesis para la implementación del primer esquema de cifrado autenticado con datos asociados basado en el cifrado de bloque ligero \ac{GFRX} \cite{zhang2023gfrx} \cite{zhang2024gfspx} utilizando el modo \ac{COFB} \cite{chakraborti2020blockcipher}. El trabajo se enfoca en desarrollar una solución optimizada para dispositivos \ac{IoT} con recursos limitados, donde los esquemas criptográficos tradicionales resultan computacionalmente costosos.

\section{Motivación y Contexto}

El \ac{IoT} representa un paradigma tecnológico en expansión exponencial. Análisis de mercado recientes estiman que el número de dispositivos IoT activos, que fue de 16.6 mil millones en 2023, crecerá hasta alcanzar los 41.1 mil millones para el año 2030, como se ilustra en la Figura~\ref{fig:iot_growth_statista} \cite{IoTAnalytics2025_report}. Estos dispositivos, caracterizados por recursos computacionales extremadamente limitados, requieren soluciones criptográficas que garanticen tanto confidencialidad como autenticidad de los datos transmitidos, sin comprometer su funcionamiento eficiente \cite{yalli2024internet}.

\begin{figure}[h!]
    \centering
    \includegraphics[width=0.9\textwidth]{figs/grafico_iot_statista_2024.png}
    \caption{Proyección de Dispositivos IoT Conectados \cite{IoTAnalytics2025_report}}
    \label{fig:iot_growth_statista}
\end{figure}

En las ciudades inteligentes, se emplean sensores para gestionar el tráfico, vigilar la calidad del aire o reforzar la seguridad pública. En el ámbito de la salud, los dispositivos \ac{IoT} permiten monitorear a los pacientes a distancia, ofrecer diagnósticos más rápidos y personalizar tratamientos. En la industria, estas tecnologías impulsan la automatización, el mantenimiento predictivo y una mayor eficiencia operativa \cite{houssein2024internet}.

Sin embargo, este crecimiento masivo también trae nuevos desafíos en materia de seguridad. Proteger la información personal de los pacientes, asegurar la infraestructura urbana y garantizar la integridad de procesos industriales se vuelve una prioridad crítica \cite{alsheavi2025iot}. Los dispositivos \ac{IoT} suelen estar conectados a redes abiertas o públicas, lo que los hace vulnerables a ataques. Un fallo de seguridad puede significar desde la filtración de datos hasta el acceso no autorizado a sistemas vitales \cite{houssein2024internet}\cite{cetintav2025review}.

Los esquemas tradicionales de \ac{AEAD}, como \ac{AES}-\ac{GCM}, aunque seguros, resultan computacionalmente costosos para dispositivos con restricciones severas de área y energía \cite{dworkin2007recommendation}. \ac{AES}-\ac{GCM} requiere aproximadamente 3175 \ac{LUT}s en implementaciones \ac{FPGA}, lo cual representa una sobrecarga significativa para sensores \ac{IoT} que típicamente disponen de menos de 5000 \ac{LUT}s totales \cite{dalmasso2019evaluation} \cite{alenezi2024performance}.

{\sloppy
La criptografía ligera surge como respuesta a estas limitaciones \cite{mohd2018lightweight}\cite{thakor2021lightweight}, priorizando la eficiencia en hardware y software con restricciones severas. \ac{GFRX} representa un cifrado de bloque ligero innovador \cite{zhang2023gfrx} que combina una estructura Feistel generalizada con operaciones \ac{ARX}. Sus características principales incluyen difusión acelerada que alcanza el efecto avalancha completo en solo 6 rondas. Además, ofrece flexibilidad de serialización completa y estructura optimizada para hardware con consumo mínimo de recursos.
\par}

\ac{GFRX} nunca ha sido implementado en un esquema \ac{AEAD}, representando una oportunidad única para desarrollar una solución ligera para \ac{IoT}.

\section{Planteamiento del Problema}

Los esquemas actuales de cifrado autenticado con datos asociados están basados en cifrados de bloque convencionales como \ac{AES}, requiriendo una considerable cantidad de recursos computacionales que los hace poco adecuados para dispositivos con restricciones severas \cite{daemen1999aes}. Estos dispositivos presentan limitaciones de hardware con menos de 10KB de \ac{RAM} y procesadores de 8-16 bits, restricciones de consumo energético debido a la operación con baterías durante años, y necesidad de seguridad robusta para el manejo de información sensible \cite{houssein2024internet}. Es necesario desarrollar soluciones criptográficas compatibles con esquemas de autenticación y cifrado, basadas en cifrados de bloque ligeros adaptados específicamente a dispositivos \ac{IoT} \cite{soto2024survey}.

\section{Objetivos}

El objetivo general de esta tesis es el siguiente.

Desarrollar y evaluar la primera implementación funcional en lenguaje C del esquema de cifrado autenticado con datos asociados \ac{GFRX}+\ac{COFB}, obteniendo métricas de rendimiento en software y estimaciones de características hardware mediante herramientas de síntesis de alto nivel, para dispositivos \ac{IoT} con recursos limitados \cite{zhang2023gfrx} \cite{chakraborti2020blockcipher}.

\subsection{Objetivos Específicos}

Para alcanzar el objetivo general, se han definido los siguientes objetivos específicos.

\begin{enumerate}
    \item Implementar en lenguaje C el cifrado de bloque ligero GFRX-128 y el modo \ac{COFB} manteniendo un estado mínimo de $1.5n+k$ bits, donde $n$ es el tamaño de bloque (128 bits) y $k$ el tamaño de clave (128 bits), resultando en 320 bits totales, con código modular y documentado que sirva como referencia funcional para futuras implementaciones en hardware.

    \item Validar la correctitud funcional de la implementación mediante vectores de prueba exhaustivos, medir el rendimiento en software mediante herramientas de profiling (Valgrind, gprof), y utilizar herramientas open-source de síntesis de alto nivel para obtener estimaciones iniciales de métricas de hardware como \ac{LUT}s, registros, frecuencia máxima y consumo energético aproximado.

    \item Analizar la resistencia teórica a ataques criptográficos del esquema hasta $2^{64}$ consultas y realizar una comparación cualitativa con esquemas \ac{AEAD} estándar como AES-GCM, ASCON y GIFT-COFB, considerando las diferencias metodológicas entre implementaciones reales en \ac{FPGA} y estimaciones basadas en síntesis de alto nivel \cite{banik2020gift}\cite{dobraunig2021ascon}.
\end{enumerate}

\section{Organización de la tesis}

El restante de este documento de tesis está organizado de la siguiente forma.

El Capítulo~\ref{chap:background} presenta el estado del arte y marco teórico sobre criptografía ligera, cifrados de bloque, esquemas \ac{AEAD}, el cifrado \ac{GFRX} y el modo \ac{COFB}.

El Capítulo~\ref{chap:proposal} describe la propuesta de integración GFRX+COFB, incluyendo la arquitectura del sistema, metodología de implementación y enfoque de validación.

El Capítulo~\ref{chap:results} presenta la implementación en lenguaje C del esquema propuesto, la metodología de validación empleada y los resultados experimentales obtenidos incluyendo mediciones de rendimiento, análisis comparativo con esquemas AEAD establecidos y evaluación de propiedades de seguridad.

\section{Cronograma}

\begin{figure}[h!]
    \centering
    \includegraphics[width=0.6\textwidth]{figs/cronograma.png}
    \caption{Mi cronograma}
    \label{fig:crocro}
\end{figure} %Inserta el capítulo 1: Introducción
\chapter{Marco Teórico}
\label{chap:background}

Este capítulo presenta el marco teórico fundamental para comprender la propuesta de esta tesis. Se inicia con una revisión del estado del arte en criptografía ligera, explorando los principales cifrados de bloque y modos de operación desarrollados para dispositivos con recursos limitados. Posteriormente, se desarrollan los conceptos fundamentales de cifrado autenticado con datos asociados \ac{AEAD}, el cifrado de bloque \ac{GFRX} y el modo \ac{COFB}, que constituyen los pilares teóricos de nuestra propuesta. Finalmente, se presentan las métricas de evaluación que permitirán validar y comparar la implementación propuesta.

\section{Criptografía Ligera -- Estado del Arte}

Esta sección presenta una revisión del desarrollo histórico y el estado actual de la criptografía ligera, destacando los principales cifrados de bloque y modos de operación diseñados específicamente para dispositivos con recursos limitados.

\subsection{Evolución de la Criptografía Ligera}

La criptografía ligera surge como respuesta a las limitaciones de recursos en dispositivos embebidos y del Internet de las Cosas \cite{yalli2024internet}. A diferencia de los algoritmos criptográficos tradicionales diseñados para computadoras de propósito general, la criptografía ligera prioriza la eficiencia en hardware y software con restricciones severas \cite{soto2024survey}\cite{thakor2021lightweight}.

\subsection{Cifrados de Bloque Ligeros Relevantes}

Entre los cifrados de bloque ligeros más destacados se encuentran los siguientes.

\begin{itemize}
    \item \textbf{PRESENT}, es uno de los primeros cifrados ligeros, basado en red \ac{SPN} con bloques de 64 bits \cite{bogdanov2007present}.
    \item \textbf{SIMON/SPECK}, forman una familia de cifrados creada por la NSA, optimizados para hardware y software respectivamente \cite{cryptoeprint:2013/404}.
    \item \textbf{GIFT}, representa una mejora de PRESENT con mejor difusión y menor costo de implementación \cite{banik2017gift}\cite{banik2020gift}.
    \item \textbf{ASCON}, ganador del concurso CAESAR en 2016 para AEAD ligero, está basado en permutación esponja \cite{dobraunig2016ascon}\cite{dobraunig2021ascon}.
    \item \textbf{GFRX}, es un cifrado innovador que combina estructura Feistel generalizada con operaciones ARX \cite{zhang2023gfrx}.
\end{itemize}

\subsection{Modos AEAD para Criptografía Ligera}

Complementando los cifrados de bloque ligeros, se han desarrollado diversos modos de operación AEAD optimizados para minimizar el consumo de recursos \cite{mcgrew2008interface}. La Tabla~\ref{tab:aead_modes} compara los principales modos AEAD ligeros según sus características de estado y procesamiento. En esta tabla, $n$ representa el tamaño de bloque en bits, $k$ el tamaño de clave en bits, y $r$ la tasa de absorción en construcciones esponja.

\begin{table}[h!]
\centering
\caption{Comparación de modos AEAD ligeros}
\label{tab:aead_modes}
\small
\begin{tabular}{|l|c|c|p{3.5cm}|}
\hline
\textbf{Modo} & \textbf{Estado (bits)} & \textbf{Tasa} & \textbf{Características} \\
\hline
JAMBU \cite{wu2016jambu} & $1.5n$ & 1 bloque & Estado mínimo \\
\hline
COFB \cite{chakraborti2020blockcipher} & $1.5n+k$ & 1 bloque & Retroalimentación combinada \\
\hline
OCB \cite{krovetz2021design} & $2n+k$ & 1 bloque & Paralelizable, patentado \\
\hline
Esponja (ASCON) \cite{sonmez2024ascon} & Variable & r bits & Hash y cifrado integrado \\
\hline
\end{tabular}
\end{table}

\section{Cifrado Autenticado con Datos Asociados}

El cifrado autenticado con datos asociados representa uno de los pilares fundamentales de la seguridad moderna en comunicaciones. Esta sección establece la definición formal y las propiedades de seguridad esenciales que debe cumplir cualquier esquema AEAD robusto \cite{krovetz2011software}.

\subsection{Definición Formal}

Un esquema AEAD es una tupla $(\mathcal{K}, \mathcal{E}, \mathcal{D})$ que incluye los siguientes elementos.

\begin{itemize}
    \item $\mathcal{K}$ representa el espacio de claves.
    \item $\mathcal{E}$ es la función de cifrado definida como $\mathcal{K} \times \mathcal{N} \times \mathcal{A} \times \mathcal{M} \rightarrow \mathcal{C} \times \mathcal{T}$.
    \item $\mathcal{D}$ es la función de descifrado definida como $\mathcal{K} \times \mathcal{N} \times \mathcal{A} \times \mathcal{C} \times \mathcal{T} \rightarrow \mathcal{M} \cup \{\bot\}$.
\end{itemize}

Donde $\mathcal{N}$ es el espacio de nonces, $\mathcal{A}$ los datos asociados, $\mathcal{M}$ el espacio de mensajes, $\mathcal{C}$ el espacio de textos cifrados, $\mathcal{T}$ las etiquetas de autenticación, y $\bot$ indica fallo de verificación \cite{rogaway2002authenticated}.

\subsection{Propiedades de Seguridad}

Un esquema AEAD seguro debe garantizar simultáneamente confidencialidad e integridad \cite{jimale2022authenticated}. A continuación se describen las dos propiedades fundamentales.

\begin{table}[h!]
\centering
\caption{Propiedades de seguridad AEAD}
\label{tab:aead_security}
\footnotesize
\begin{tabular}{|p{3cm}|p{8cm}|}
\hline
\textbf{Propiedad} & \textbf{Descripción} \\
\hline
\ac{IND-CPA} (Confidencialidad) & Un adversario no puede distinguir entre el cifrado de dos mensajes elegidos con probabilidad significativamente mayor a 1/2 \cite{bellare2000authenticated}. \\
\hline
\ac{INT-CTXT} (Autenticidad) & La probabilidad de que un adversario genere un texto cifrado y tag válidos sin conocer la clave es despreciable, típicamente limitada por el tamaño del tag \cite{bellare2000authenticated}. \\
\hline
\end{tabular}
\end{table}

\section{Cifrado de Bloque GFRX}

GFRX es un cifrado de bloque ligero de reciente desarrollo que representa una innovación significativa en el diseño de primitivas criptográficas para dispositivos con recursos limitados. Esta sección detalla su estructura, parámetros y propiedades de seguridad \cite{zhang2023gfrx}.

\subsection{Estructura y Parámetros}

GFRX emplea una estructura Feistel generalizada de 4 ramas con funciones de ronda basadas en ARX. El estado se divide en cuatro palabras que se procesan mediante las funciones FAN (basada en AND/XOR con rotaciones) y FAD (basada en ADD con desplazamientos). Estas funciones no lineales se combinan con operaciones XOR utilizando subclaves derivadas de la clave maestra.

La Tabla~\ref{tab:gfrx_params} presenta los parámetros principales de la variante GFRX-128/128 utilizada en esta propuesta.

\begin{table}[h!]
\centering
\caption{Parámetros de GFRX-128/128}
\label{tab:gfrx_params}
\begin{tabular}{|l|c|}
\hline
\textbf{Parámetro} & \textbf{Valor} \\
\hline
Tamaño de bloque & 128 bits \\
\hline
Tamaño de clave & 128 bits \\
\hline
Número de rondas & 32 \\
\hline
Operaciones por ronda & 3 AND/ADD, 4 ROT, 5 XOR \\
\hline
Estructura & Feistel generalizada (4 ramas) \\
\hline
Difusión completa & 6 rondas \\
\hline
\end{tabular}
\end{table}

\subsection{Análisis de Seguridad}

La Tabla~\ref{tab:gfrx_security} resume el análisis de seguridad de GFRX-128/128 según el autor \cite{zhang2023gfrx}, mostrando un margen de seguridad superior al 40\% respecto a los mejores ataques conocidos.

\begin{table}[h!]
\centering
\caption{Análisis de seguridad de GFRX-128/128}
\label{tab:gfrx_security}
\begin{tabular}{|l|c|c|}
\hline
\textbf{Tipo de ataque} & \textbf{Rondas atacadas} & \textbf{Margen de seguridad} \\
\hline
Diferencial & 19 rondas & 13 rondas (40.6\%) \\
\hline
Lineal & 13 rondas & 19 rondas (59.4\%) \\
\hline
Total de rondas & \multicolumn{2}{c|}{32 rondas} \\
\hline
\end{tabular}
\end{table}

\section{Modo COFB}

El modo COFB es un modo de operación AEAD diseñado específicamente para minimizar los requisitos de estado, haciéndolo ideal para implementaciones en hardware con recursos extremadamente limitados. Esta sección describe sus componentes fundamentales y su algoritmo de procesamiento \cite{chakraborti2020blockcipher}.

\subsection{Características Principales}

COFB se caracteriza por utilizar una función de retroalimentación combinada que mezcla el texto plano con la salida del cifrado de bloque mediante operaciones XOR y una transformación lineal invertible. Esta función permite generar tanto el texto cifrado como la entrada para la siguiente invocación del cifrado de bloque de manera eficiente.

El modo utiliza máscaras de 64 bits (la mitad del tamaño del bloque) generadas mediante multiplicación en el campo finito de Galois $\text{GF}(2^{64})$, lo que reduce significativamente los requisitos de estado comparado con otros modos AEAD. La Tabla~\ref{tab:cofb_features} resume las características principales del modo COFB, donde $|AD|$ denota el tamaño de los datos asociados en bits y $|M|$ el tamaño del mensaje en bits.

\begin{table}[h!]
\centering
\caption{Características del modo COFB}
\label{tab:cofb_features}
\begin{tabular}{|l|c|}
\hline
\textbf{Característica} & \textbf{Valor} \\
\hline
Estado requerido & $1.5n + k$ bits \\
\hline
Tamaño de máscara & $n/2$ bits (64 bits para n=128) \\
\hline
Tasa de procesamiento & 1 bloque por invocación \\
\hline
Paralelizable & No \\
\hline
Invocaciones del cifrado & $|AD|/n + |M|/n + 2$ \\
\hline
Campo finito para máscaras & GF($2^{64}$) \\
\hline
\end{tabular}
\end{table}

El procesamiento COFB se realiza en tres fases principales. Primero se inicializa el estado con el nonce, luego se procesan los datos asociados aplicando máscaras específicas, y finalmente se cifra el mensaje generando simultáneamente el texto cifrado y el tag de autenticación. Los detalles algorítmicos completos pueden consultarse en la especificación original \cite{chakraborti2020blockcipher}.

\section{Métricas de Evaluación}

Para evaluar objetivamente el rendimiento de cualquier implementación criptográfica en hardware, es fundamental establecer un conjunto de métricas cuantitativas. Esta sección define las métricas que se utilizarán para caracterizar la implementación propuesta.

\subsection{Métricas de Hardware}
Para evaluar el rendimiento hardware se considerarán las siguientes métricas siguiendo el framework de evaluación establecido \cite{konstantopoulou2025review}:
\begin{itemize}
    \item \textbf{Área} medida en LUTs, Slices y Registros
    \item \textbf{Velocidad} evaluada mediante Throughput en Gbps y Latencia en ciclos por bloque
    \item \textbf{Eficiencia} calculada como Mbps/LUT y Eficiencia energética en mW/Gbps
\end{itemize}

\subsection{Métricas de Seguridad}
Para el análisis de seguridad se utilizarán las siguientes métricas basadas en estudios previos \cite{sun2021accelerating}\cite{khairallah2022security}:
\begin{itemize}
    \item Resistencia a falsificación hasta $2^{64}$ consultas
    \item Análisis de colisiones en tags
    \item Evaluación de reutilización de nonces
\end{itemize}

\section{Consideraciones Finales}

Este capítulo ha establecido el marco teórico de la criptografía ligera, presentando el cifrado \ac{GFRX} \cite{zhang2023gfrx} con su estructura Feistel generalizada que alcanza difusión completa en 6 rondas, el modo \ac{COFB} \cite{chakraborti2020blockcipher} con su diseño minimalista de $1.5n+k$ bits de estado, y las métricas de evaluación que permitirán comparar nuestra propuesta con esquemas establecidos como AES-GCM, ASCON y GIFT-COFB.

Con esta base teórica establecida, el siguiente capítulo presenta la propuesta de integración GFRX+COFB, detallando la arquitectura del sistema, la metodología de implementación y los resultados esperados.
 %Inserta el capítulo 2: Marco Teórico
\chapter{Propuesta}
\label{chap:proposal}

\section{Introducción a la Propuesta}

Esta tesis propone la primera implementación del modo COFB \cite{chakraborti2020blockcipher} utilizando GFRX \cite{zhang2023gfrx} como primitiva base de cifrado. COFB, es un modo AEAD que minimiza el tamaño de estado a $1.5n+k$ bits, donde $n$ es el tamaño del bloque y $k$ el tamaño de la clave.

La combinación GFRX+COFB promete ofrecer una solución ligera para dispositivos IoT con las siguientes características esperadas. La Figura~\ref{fig:arquitectura_gfrx_cofb} diagrama la forma en que COFB actúa como el controlador lógico que utiliza la primitiva de cifrado GFRX como motor.

\begin{figure}[htbp!]
    \centering
    \includegraphics[width=0.9\textwidth]{figs/arquitectura_gfrx_cofb.png}
    \caption{Arquitectura de alto nivel del sistema AEAD GFRX+COFB propuesto.}
    \label{fig:arquitectura_gfrx_cofb}
\end{figure}

\section{Arquitectura General del Sistema}

Esta sección describe la estructura modular del sistema GFRX+COFB propuesto, detallando cada uno de sus componentes principales y el flujo de procesamiento que seguirá la información a través del sistema.

\subsection{Componentes Principales}

La arquitectura propuesta de GFRX+COFB se organizará en cinco módulos principales.

\begin{enumerate}
    \item \textbf{GFRX\_Core} será el núcleo que implementará el cifrado de bloque GFRX-128 con 32 rondas. Este módulo será el componente más crítico en términos de área y velocidad.

    \item \textbf{COFB\_Controller} consistirá en una máquina de estados finitos FSM que controlará el flujo de procesamiento AEAD, gestionando las diferentes fases como inicialización, procesamiento de datos asociados, cifrado de mensaje y generación/verificación de tag.

    \item \textbf{Feedback\_Function} implementará las funciones $\rho$ y $\rho^{-1}$ para manejar la retroalimentación combinada tanto en modo cifrado como descifrado.

    \item \textbf{Mask\_Generator} generará y actualizará las máscaras de 64 bits utilizando multiplicación en $\text{GF}(2^{64})$ según el contexto de AD o mensaje.

    \item \textbf{Interface\_Module} proporcionará la interfaz estándar para operaciones AEAD \cite{mcgrew2008interface}, compatible con las especificaciones de criptografía ligera del \ac{NIST} \cite{nist2023lightweight}.
\end{enumerate}

\subsection{Flujo de Procesamiento}

El sistema seguirá un flujo secuencial de procesamiento que comprende cuatro etapas principales. La Tabla~\ref{tab:processing_flow} describe cada etapa y sus entradas/salidas.

\begin{table}[h!]
\centering
\caption{Flujo de procesamiento GFRX+COFB}
\label{tab:processing_flow}
\begin{tabular}{|l|l|l|}
\hline
\textbf{Etapa} & \textbf{Entrada} & \textbf{Salida} \\
\hline
Inicialización & Nonce (N), Clave (K) & Estado inicial \\
\hline
Procesamiento AD & Datos asociados (A) & Estado actualizado \\
\hline
Procesamiento Mensaje & Mensaje (M) & Texto cifrado (C) \\
\hline
Generación Tag & Estado final & Tag de autenticación (T) \\
\hline
\end{tabular}
\end{table}

\section{Metodología de Diseño}

Esta sección presenta las decisiones fundamentales de diseño que guiarán la implementación del sistema, incluyendo el enfoque arquitectural, las estrategias de optimización y los detalles específicos de cada componente crítico.

\subsection{Enfoque de Diseño}

Se propone una metodología de diseño en dos etapas. Primero, se implementará el sistema completo en lenguaje C, lo que permitirá validación funcional rápida y medición de métricas de software. Segundo, se utilizarán herramientas de síntesis de alto nivel para obtener estimaciones de las características de hardware sin necesidad de implementación manual en HDL. Esta decisión de diseño balanceará los siguientes aspectos.

\begin{itemize}
    \item \textbf{Velocidad de desarrollo} mediante implementación directa en C
    \item \textbf{Validación funcional} rápida con herramientas estándar de software
    \item \textbf{Estimación de hardware} mediante simuladores de síntesis de alto nivel
    \item \textbf{Optimización iterativa} entre implementación C y proyección hardware
\end{itemize}

\subsection{Diseño del Núcleo GFRX-128}

El núcleo GFRX se implementará en C siguiendo la estructura Feistel generalizada de 4 ramas con las funciones FAN y FAD. Los componentes clave incluirán los siguientes elementos.

\begin{itemize}
    \item Funciones FAN y FAD implementadas como operaciones bit a bit en C
    \item Estado representado mediante arreglos de enteros de 32 bits
    \item Bucle de rondas parametrizable para facilitar análisis de seguridad
    \item Módulo de expansión de clave eficiente usando rotaciones y XOR
\end{itemize}

\subsection{Implementación del Modo COFB}

La implementación del modo COFB en C requerirá los siguientes componentes.

\begin{itemize}
    \item Función lineal invertible $G$ implementada mediante operaciones XOR
    \item Generador de máscaras basado en multiplicación en $\text{GF}(2^{64})$
    \item Funciones de padding para bloques incompletos
    \item Gestión de contexto para diferenciar procesamiento de AD y mensaje
\end{itemize}

\section{Metodología de Implementación}

Esta sección detalla el plan de implementación del sistema propuesto, especificando las herramientas que se utilizarán y las fases secuenciales en las que se estructurará el desarrollo del proyecto.

\subsection{Herramientas de Desarrollo}

La Tabla~\ref{tab:dev_tools} presenta las herramientas de desarrollo que se utilizarán para la implementación en software, análisis de rendimiento y estimación de características hardware.

\begin{table}[h!]
\centering
\caption{Herramientas de desarrollo}
\label{tab:dev_tools}
\begin{tabular}{|l|l|p{5cm}|}
\hline
\textbf{Categoría} & \textbf{Herramienta} & \textbf{Propósito} \\
\hline
Lenguaje & C (C99/C11) & Implementación funcional \\
\hline
Compilador & GCC / Clang & Compilación y optimización \\
\hline
Profiling & Valgrind / gprof & Análisis de rendimiento software \\
\hline
Síntesis \ac{HLS} & LegUp / Bambu \ac{HLS} & Estimación open-source de métricas hardware \\
\hline
Testing & CUnit / Google Test & Framework de pruebas unitarias \\
\hline
\end{tabular}
\end{table}

\subsection{Fases de Implementación}

La Tabla~\ref{tab:implementation_phases} detalla las cuatro fases de implementación del proyecto, sus componentes principales y duración estimada.

\begin{table}[h!]
\centering
\caption{Fases de implementación del proyecto}
\label{tab:implementation_phases}
\begin{tabular}{|l|p{7cm}|}
\hline
\textbf{Fase} & \textbf{Componentes}\\
\hline
Fase 1: GFRX en C & Funciones de ronda, expansión de claves, cifrado completo de 32 rondas \\
\hline
Fase 2: COFB en C & Función de retroalimentación, generador de máscaras $\text{GF}(2^{64})$, interfaz AEAD \\
\hline
Fase 3: Validación & Vectores de prueba, validación funcional, profiling de rendimiento software \\
\hline
Fase 4: Síntesis \ac{HLS} & Síntesis de alto nivel, estimación de área, frecuencia y potencia en hardware \\
\hline
\end{tabular}
\end{table}

\section{Metodología de Validación}

La validación del sistema es fundamental para garantizar su correctitud funcional. Esta sección describe la estrategia multinivel de validación que se aplicará, los tipos de vectores de prueba que se generarán y el rol de la implementación de referencia en software.

\subsection{Validación Funcional}

La Tabla~\ref{tab:validation_levels} describe los tres niveles de validación funcional que se aplicarán para garantizar la correctitud del sistema.

\begin{table}[h!]
\centering
\caption{Niveles de validación funcional}
\label{tab:validation_levels}
\begin{tabular}{|l|p{6cm}|p{4cm}|}
\hline
\textbf{Nivel} & \textbf{Alcance} & \textbf{Módulos objetivo} \\
\hline
Componente & Pruebas unitarias individuales & GFRX\_Core, Feedback\_Function, Mask\_Generator \\
\hline
Integración & Pruebas de interfaces entre módulos & Todos los módulos integrados \\
\hline
Sistema & Pruebas end-to-end completas & Sistema GFRX+COFB completo \\
\hline
\end{tabular}
\end{table}

\subsection{Vectores de Prueba}

Se generarán vectores de prueba exhaustivos que incluirán los siguientes tipos.

\begin{itemize}
    \item Vectores básicos con mensajes y AD de longitud variable
    \item Casos extremos considerando longitud cero, bloques incompletos y múltiples bloques
    \item Vectores de regresión formando un conjunto estándar para verificación continua
    \item Vectores aleatorios generados para pruebas de robustez
\end{itemize}

\subsection{Implementación de Referencia}

La implementación en C servirá simultáneamente como implementación funcional y de referencia para los siguientes propósitos.

\begin{itemize}
    \item Generar vectores de prueba confiables mediante ejecución directa
    \item Validar la correctitud mediante comparación con especificaciones
    \item Realizar análisis de cobertura de código mediante herramientas de profiling
    \item Servir como entrada para herramientas de síntesis de alto nivel
\end{itemize}

\section{Metodología de Evaluación}

Para caracterizar el desempeño del sistema propuesto de manera objetiva y completa, es esencial definir un conjunto de métricas cuantitativas y realizar comparaciones con esquemas existentes. Esta sección establece las métricas de hardware y seguridad que se utilizarán, así como el marco de comparación con otros esquemas AEAD.

\subsection{Métricas de Evaluación}

La evaluación considerará tanto métricas de software como estimaciones de hardware. Las métricas de evaluación para implementaciones de criptografía ligera deben considerar múltiples dimensiones de rendimiento \cite{mckay2016report}. Siguiendo las directrices de \ac{NIST}, definimos métricas tanto de hardware como de software. La Tabla~\ref{tab:metricas_evaluacion_detalladas} presenta las métricas que se medirán o estimarán.

\begin{table}[htbp]
\centering
\caption{Métricas de evaluación basadas en framework ATHENa y estándares \ac{NIST}}
\label{tab:metricas_evaluacion_detalladas}
\begin{tabular}{@{}lll@{}}
\toprule
\textbf{Categoría} & \textbf{Métrica} & \textbf{Método} \\
\midrule
\multirow{3}{*}{Software}
& Throughput (Mbps) & Medición directa \\
& Ciclos/bloque & Profiling (gprof) \\
& Memoria (KB) & Runtime analysis \\
\midrule
\multirow{5}{*}{\begin{tabular}[c]{@{}l@{}}Hardware\\(\ac{HLS} estimado)\end{tabular}}
& LUTs/Slices & Síntesis \ac{HLS} \\
& Registros (FF) & Reportes \ac{HLS} \\
& Frecuencia (MHz) & Timing analysis \\
& Latencia (ciclos) & Simulación \ac{HLS} \\
& Throughput-to-Area & Mbps/LUT \\
\midrule
\multirow{4}{*}{Seguridad}
& Resistencia forgery & Análisis teórico \\
& Birthday bound & $2^{n/2}$ consultas \\
& Límites de uso & Consultas seguras \\
& Margen seguridad & Rondas adicionales \\
\bottomrule
\end{tabular}
\end{table}

{\sloppy
Dado que utilizamos síntesis de alto nivel (\ac{HLS}) para obtener estimaciones de características hardware, justificamos este enfoque basándonos en estudios previos \cite{homsirikamol2014can}\cite{homsirikamol2015hardware}. Estos demuestran que el ranking relativo de algoritmos permanece consistente al comparar \ac{HLS} y \ac{HDL} manual.
\par}

\subsection{Métricas de Seguridad}

El análisis de seguridad de GFRX+COFB considera múltiples vectores de ataque. Para el modo COFB específicamente, analizamos la resistencia a ataques de falsificación siguiendo la metodología de Khairallah \cite{khairallah2022security} e Inoue y Minematsu \cite{inoue2021gift}.
Evaluamos las siguientes propiedades de seguridad AEAD \cite{cremers2023automated}.
\begin{itemize}
\item Confidencialidad (\ac{IND-CPA})
\item Integridad (\ac{INT-CTXT})  
\item Resistencia a nonce-misuse
\item Límites de uso seguros \cite{gunther2023usage}
\end{itemize}

\subsection{Comparación con Esquemas Existentes}

Se realizará una comparación con esquemas AEAD representativos del estado del arte siguiendo el proceso de evaluación de \ac{NIST} \cite{nist2023lightweight}. Los esquemas seleccionados son:

\begin{itemize}
\item \textbf{AES-GCM} \cite{dworkin2007recommendation}: Estándar actual para aplicaciones generales
\item \textbf{ASCON} \cite{sonmez2024ascon}: Ganador del concurso \ac{NIST} LWC 2023
\item \textbf{GIFT-COFB} \cite{inoue2021gift}: Esquema comparable usando COFB como modo de operación
\item \textbf{TinyJAMBU} \cite{wu2019tinyjambu}: Finalista NIST LWC, modo ultra-ligero
\end{itemize}

La Tabla~\ref{tab:esquemas_aead_comparacion} presenta una comparación de las características principales de estos esquemas basada en implementaciones reportadas en el benchmark ATHENa \cite{mohajerani2020fpga} y análisis recientes \cite{konstantopoulou2025review}.

\begin{table}[h!]
\centering
\caption{Comparación de esquemas AEAD (valores aproximados de literatura)}
\label{tab:esquemas_aead_comparacion}
\small
\begin{tabular}{|l|l|c|c|c|l|}
\hline
\textbf{Esquema} & \textbf{Primitiva} & \textbf{Estado} & \textbf{LUTs} & \textbf{Throughput} & \textbf{Tipo} \\
 & & \textbf{(bits)} & \textbf{(aprox)} & \textbf{@ 100MHz} & \textbf{Impl.} \\
\hline
AES-GCM & AES-128 & 384 & $\sim$3175 & 1.28 Gbps & FPGA Real \\
\hline
ASCON & Permutación & 320 & $\sim$1712 & 640 Mbps & FPGA Real \\
\hline
GIFT-COFB & GIFT-128 & 320 & $\sim$1450 & 400 Mbps & FPGA Real \\
\hline
TinyJAMBU & TJ-128 & 288 & $\sim$800 & 200 Mbps & FPGA Real \\
\hline
\textbf{GFRX+COFB} & \textbf{GFRX-128} & \textbf{320} & \textbf{A medir} & \textbf{A medir} & \textbf{\ac{HLS} Est.} \\
\textbf{(propuesto)} & & & & & \\
\hline
\end{tabular}
\end{table}

\textbf{Nota importante:} Los valores de AES-GCM, ASCON, GIFT-COFB y TinyJAMBU corresponden a implementaciones FPGA reales reportadas en literatura científica. Los valores de GFRX+COFB se obtendrán mediante la implementación y síntesis \ac{HLS} propuesta en esta tesis.

La comparación evaluará área de hardware (LUTs y Slices), throughput y eficiencia (Mbps/LUT), consumo energético, nivel de seguridad teórico y estado requerido, siguiendo las prácticas establecidas \cite{jimale2022authenticated}.

\section{Resultados Esperados}

Se espera que la implementación GFRX+COFB propuesta alcance los siguientes resultados, reconociendo que las métricas de hardware son estimaciones preliminares sujetas a las limitaciones inherentes a las herramientas de síntesis de alto nivel.

\begin{enumerate}
    \item \textbf{Implementación funcional en C} completa y validada con vectores de prueba exhaustivos, que servirá como referencia para futuras implementaciones en hardware.

    \item \textbf{Métricas de rendimiento en software} medidas mediante profiling con herramientas estándar en procesadores embebidos típicos (ARM Cortex-M, RISC-V) \cite{krovetz2011software}.

    \item \textbf{Estimaciones de área hardware} obtenidas mediante herramientas open-source de síntesis \ac{HLS}, que permitirán una comparación cualitativa con esquemas como GIFT-COFB \cite{banik2020gift}, reconociendo un margen de incertidumbre respecto a implementaciones \ac{RTL} optimizadas manualmente.

    \item \textbf{Estado mínimo} de 320 bits totales ($1.5n+k$ con $n=128, k=128$) verificado en la implementación, representando uno de los requisitos de estado más bajos entre esquemas AEAD de 128 bits.

    \item \textbf{Estimaciones de throughput hardware} obtenidas mediante síntesis \ac{HLS}, dependiendo significativamente del nivel de optimización logrado y las características de la herramienta utilizada.

    \item \textbf{Seguridad teórica robusta} heredada de GFRX (margen mayor al 40\% contra ataques diferenciales y lineales) \cite{zhang2023gfrx} y del modo COFB (seguridad probada hasta $2^{64}$ consultas) \cite{chakraborti2020blockcipher}, sin implementación práctica de ataques de canal lateral.

    \item \textbf{Estimaciones de consumo energético} obtenidas mediante herramientas \ac{HLS}, reconociendo que estas estimaciones pueden variar significativamente en implementaciones FPGA reales.
\end{enumerate}

{\sloppy
Los resultados obtenidos permitirán evaluar si GFRX+COFB puede posicionarse como una alternativa competitiva para dispositivos IoT con severas restricciones de recursos \cite{mohd2018lightweight}. La validación definitiva de estas características requeriría una implementación \ac{RTL} completa y pruebas en hardware real, lo cual queda fuera del alcance de esta tesis pero se propone como trabajo futuro.
\par}

\section{Consideraciones Finales}

Este capítulo ha presentado la propuesta completa para la implementación del primer esquema GFRX+COFB, incluyendo una arquitectura de cinco módulos principales, una metodología de diseño basada en implementación en C seguida de síntesis \ac{HLS}, y un plan de cuatro fases con tres niveles de validación funcional. El estado mínimo de 320 bits y las métricas de evaluación definidas permitirán comparar la propuesta con esquemas establecidos como AES-GCM, ASCON y GIFT-COFB.

Es importante reconocer las limitaciones inherentes: las estimaciones \ac{HLS} son preliminares y pueden diferir de implementaciones \ac{RTL} optimizadas, el análisis de seguridad se fundamenta en las propiedades demostradas en las publicaciones originales sin implementación práctica de ataques, y no se contempla validación en hardware real. Los resultados esperados incluyen una implementación funcional en C validada exhaustivamente con métricas de software y estimaciones hardware preliminares.

Con la arquitectura y metodología establecidas, el siguiente paso será ejecutar la implementación para validar experimentalmente las características de GFRX+COFB y evaluar su viabilidad como solución criptográfica ligera para dispositivos \ac{IoT}.
 %Inserta el capítulo 3: Propuesta
\chapter{Pruebas y Resultados}
\label{chap:results}

Este capítulo presenta la implementación del esquema GFRX+COFB y los resultados experimentales obtenidos. Se describe primero la implementación en C99, las decisiones de diseño y la metodología de validación empleada. Posteriormente se presentan los resultados de validación funcional, mediciones de rendimiento, análisis comparativo con esquemas AEAD establecidos y evaluación de propiedades de seguridad.

\section{Implementación}

\subsection{Visión General}

La implementación de GFRX+COFB se desarrolló en lenguaje C siguiendo el estándar C99, priorizando la portabilidad, eficiencia y claridad del código. La implementación se estructuró en tres módulos principales con 561 líneas de código totales.

\begin{itemize}
    \item \textbf{Módulo GFRX} en el archivo \texttt{gfrx.c} de 146 líneas implementa el cifrado de bloque GFRX-128/128 con sus funciones de ronda y expansión de clave.
    \item \textbf{Módulo COFB} en el archivo \texttt{cofb.c} de 397 líneas implementa el modo de operación COFB incluyendo procesamiento de datos asociados, cifrado/descifrado de mensajes y generación/verificación de tags.
    \item \textbf{Módulo de Utilidades} en el archivo \texttt{utils.c} de 18 líneas proporciona funciones de seguridad críticas para comparación constante en tiempo y limpieza segura de memoria.
\end{itemize}

\subsection{Decisiones de Diseño}

Las principales decisiones de diseño tomadas se describen a continuación.

\begin{enumerate}
    \item \textbf{Lenguaje C estándar C99} fue seleccionado por su portabilidad, eficiencia y idoneidad para sistemas embebidos. El estándar C99 proporciona tipos de datos de tamaño fijo como \texttt{uint8\_t} y \texttt{uint32\_t}, esenciales para implementaciones criptográficas deterministas.

    \item \textbf{Sin dependencias externas}, la implementación núcleo no requiere bibliotecas externas. Únicamente depende de la biblioteca estándar de C con los headers \texttt{stdint.h} y \texttt{string.h}, maximizando la portabilidad entre plataformas.

    \item \textbf{Gestión de memoria basada en pila} donde toda la gestión de memoria utiliza asignación en stack. Se evita completamente asignación dinámica mediante \texttt{malloc} y \texttt{free}, eliminando fugas de memoria y haciendo la implementación adecuada para sistemas de tiempo real.

    \item \textbf{API orientada a contextos} utiliza estructuras de contexto como \texttt{gfrx\_ctx\_t} y \texttt{cofb\_ctx\_t} que encapsulan el estado interno, permitiendo múltiples instancias independientes simultáneas.
\end{enumerate}

\subsection{Parámetros Principales}

El módulo GFRX define los siguientes parámetros.

\begin{itemize}
    \item \textbf{Tamaño de bloque} de 128 bits equivalentes a 16 bytes
    \item \textbf{Tamaño de clave} de 128 bits equivalentes a 16 bytes
    \item \textbf{Número de rondas} de 32 en red Feistel
    \item \textbf{Tamaño de nonce COFB} de 64 bits equivalentes a 8 bytes
    \item \textbf{Tamaño de tag COFB} de 128 bits equivalentes a 16 bytes
\end{itemize}

La estructura de contexto \texttt{gfrx\_ctx\_t} almacena el programa de claves expandido de 128 bytes, mientras que \texttt{cofb\_ctx\_t} ocupa 144 bytes totales incluyendo el contexto GFRX y la constante L de 16 bytes.

\subsection{Métricas de Implementación}

La Tabla~\ref{tab:implementation_metrics} resume las métricas de la implementación.

\begin{table}[h!]
\centering
\caption{Métricas de la implementación GFRX+COFB}
\label{tab:implementation_metrics}
\begin{tabular}{|l|r|}
\hline
\textbf{Métrica} & \textbf{Valor} \\
\hline
Líneas de código totales & 561 LOC \\
\hline
Tamaño del código compilado & 6,778 bytes \\
\hline
Uso de RAM para mensaje 256 bytes & ~1 KB \\
\hline
Contexto COFB & 144 bytes \\
\hline
Dependencias externas & 0 \\
\hline
Asignaciones dinámicas de memoria & 0 \\
\hline
\end{tabular}
\end{table}

\section{Metodología de Validación}

\subsection{Suite de Pruebas}

Se implementó una suite de 1,666 tests automatizados que cubren todos los aspectos funcionales de GFRX+COFB. La distribución de tests se describe a continuación.

\begin{enumerate}
    \item \textbf{Pruebas básicas GFRX} con 102 tests ejecutados
    \begin{itemize}
        \item Cifrado y descifrado de bloques individuales
        \item Verificación de consistencia mediante 100 rondas de cifrado/descifrado
        \item Validación del programa de claves expandido
    \end{itemize}

    \item \textbf{Pruebas COFB modo AEAD} con 541 tests ejecutados
    \begin{itemize}
        \item Mensaje vacío como caso extremo crítico
        \item Mensajes con y sin datos asociados
        \item Longitudes variables de 0 a 512 bytes totalizando 513 tests exhaustivos
        \item Verificación correcta de autenticación
        \item Comportamiento con nonces únicos
    \end{itemize}

    \item \textbf{Pruebas de seguridad} con 23 tests ejecutados
    \begin{itemize}
        \item Detección de modificación de tag mediante 10 tests
        \item Detección de modificación de datos asociados mediante 10 tests
        \item Verificación del efecto avalancha mediante 1 test
        \item Verificación de unicidad de nonce mediante 2 tests
    \end{itemize}

    \item \textbf{Pruebas de estrés} con 1000 tests ejecutados
    \begin{itemize}
        \item 1000 operaciones consecutivas de cifrado y descifrado
        \item Verificación de estabilidad a largo plazo
        \item Detección de fugas de memoria acumuladas
    \end{itemize}
\end{enumerate}

\subsection{Herramientas de Análisis}

Se emplearon las siguientes herramientas para garantizar la calidad del código.

\begin{itemize}
    \item \textbf{Valgrind} para detección de fugas de memoria confirmando 0 bytes perdidos en 0 bloques
    \item \textbf{AddressSanitizer} para detección de errores de memoria sin desbordamientos de buffer ni accesos inválidos
    \item \textbf{UndefinedBehaviorSanitizer} para detección de comportamiento indefinido sin casos detectados
    \item \textbf{gprof} para profiling de CPU revelando distribución balanceada 50/50 entre GFRX y COFB
    \item \textbf{GCC warnings} con flags \texttt{-Wall -Wextra} sin warnings generados
\end{itemize}

\subsection{Vectores de Prueba}

Se generaron 10 vectores de prueba oficiales documentados en formato hexadecimal para garantizar reproducibilidad y permitir validación de implementaciones independientes. Los vectores cubren desde cifrado básico GFRX-128 hasta mensajes largos de 256 bytes, incluyendo casos extremos con todo-ceros y todo-unos.

\section{Resultados de Validación Funcional}

\subsection{Resultados de Tests}

La Tabla~\ref{tab:test_results_summary} muestra los resultados de la suite de pruebas.

\begin{table}[h!]
\centering
\caption{Resumen de resultados de pruebas funcionales}
\label{tab:test_results_summary}
\begin{tabular}{|l|r|r|}
\hline
\textbf{Categoría de Prueba} & \textbf{Tests Ejecutados} & \textbf{Resultado} \\
\hline
GFRX cifrado básico & 102 & 100\% aprobados \\
\hline
COFB modo AEAD & 541 & 100\% aprobados \\
\hline
Seguridad de autenticación & 23 & 100\% aprobados \\
\hline
Pruebas de estrés & 1,000 & 100\% aprobados \\
\hline
\textbf{Total} & \textbf{1,666} & \textbf{100\% aprobados} \\
\hline
\end{tabular}
\end{table}

La implementación pasó exitosamente todos los tests funcionales sin fallos, demostrando correctitud completa.

\subsection{Casos Extremos Validados}

Se verificó comportamiento correcto en casos extremos críticos.

\begin{itemize}
    \item \textbf{Mensaje vacío} de 0 bytes con tag generado correctamente
    \item \textbf{Bloques parciales} de 1-15 bytes con padding 10* aplicado correctamente
    \item \textbf{Múltiples bloques completos} de 16, 32, 64 hasta 512 bytes sin errores de encadenamiento
    \item \textbf{Datos asociados} variados con autenticación correcta con/sin AD
    \item \textbf{Patrones extremos} de todo-ceros y todo-unos procesados correctamente
\end{itemize}

\subsection{Verificación de Calidad de Código}

La Tabla~\ref{tab:code_quality} muestra los resultados de análisis de calidad.

\begin{table}[h!]
\centering
\caption{Resultados de análisis de calidad de código}
\label{tab:code_quality}
\begin{tabular}{|l|l|l|}
\hline
\textbf{Herramienta} & \textbf{Métrica} & \textbf{Resultado} \\
\hline
Valgrind & Fugas de memoria & 0 fugas detectadas \\
\hline
AddressSanitizer & Errores de memoria & 0 errores detectados \\
\hline
UBSanitizer & Comportamiento indefinido & 0 casos detectados \\
\hline
GCC warnings & Warnings de compilación & 0 warnings con flags -Wall -Wextra \\
\hline
\end{tabular}
\end{table}

La implementación está libre de errores comunes de programación y fugas de memoria.

\section{Resultados de Rendimiento}

\subsection{Plataforma de Benchmarking}

Todos los benchmarks se ejecutaron en la siguiente configuración.

\begin{itemize}
    \item \textbf{CPU} x86-64 Intel o AMD con AES-NI para AES-GCM
    \item \textbf{OS} Linux kernel 4.4.0
    \item \textbf{Compilador} GCC con \texttt{-O2 -std=c99}
    \item \textbf{Método} \texttt{clock\_gettime(CLOCK\_MONOTONIC)} con precisión de nanosegundos
    \item \textbf{Repeticiones} mínimo 1000 iteraciones por prueba con duración mayor o igual a 1 segundo
\end{itemize}

\subsection{Throughput GFRX+COFB}

La Tabla~\ref{tab:gfrx_throughput} presenta el throughput medido para diferentes tamaños de mensaje.

\begin{table}[h!]
\centering
\caption{Throughput de GFRX+COFB por tamaño de mensaje}
\label{tab:gfrx_throughput}
\begin{tabular}{|r|r|r|}
\hline
\textbf{Tamaño Mensaje bytes} & \textbf{Throughput Mbps} & \textbf{Latencia (µs)} \\
\hline
16 & 289.11 & 0.443 \\
\hline
64 & 616.90 & 0.830 \\
\hline
256 & 889.27 & 2.303 \\
\hline
1,024 & 871.33 & 9.402 \\
\hline
4,096 & 552.84 & 59.272 \\
\hline
16,384 & 220.91 & 593.337 \\
\hline
\end{tabular}
\end{table}

Los resultados muestran un pico de rendimiento de 889 Mbps para mensajes de 256 bytes. Para mensajes pequeños de 16-64 bytes se obtienen 289-617 Mbps, ideal para paquetes IoT típicos. La latencia ultra-baja de 0.443 µs para 16 bytes es crítica para IoT en tiempo real.

\subsection{Comparación con ASCON-128}

La Tabla~\ref{tab:ascon_comparison} compara GFRX+COFB con ASCON-128, ganador del concurso NIST LWC 2023.

\begin{table}[h!]
\centering
\caption{Comparación GFRX+COFB vs ASCON-128}
\label{tab:ascon_comparison}
\begin{tabular}{|r|r|r|r|}
\hline
\textbf{Mensaje} & \textbf{GFRX+COFB} & \textbf{ASCON-128} & \textbf{Ventaja} \\
\textbf{bytes} & \textbf{Mbps} & \textbf{Mbps} & \textbf{GFRX+COFB} \\
\hline
16 & 289.11 & 191.01 & +51\% de 1.51× \\
\hline
64 & 616.90 & 394.10 & +56\% de 1.56× \\
\hline
256 & 889.27 & 532.51 & +67\% de 1.67× \\
\hline
1,024 & 871.33 & 594.57 & +47\% de 1.47× \\
\hline
4,096 & 552.84 & 611.32 & -10\% de 0.90× \\
\hline
16,384 & 220.91 & 600.81 & -63\% de 0.37× \\
\hline
\end{tabular}
\end{table}

GFRX+COFB supera consistentemente a ASCON en mensajes pequeños a medianos de 16-1024 bytes, que son típicos en aplicaciones IoT. La ventaja es de 1.5× a 1.7× en el rango crítico de 16-256 bytes.

\subsection{Comparación con AES-128-GCM}

La Tabla~\ref{tab:aes_gcm_comparison} compara con el estándar actual AES-GCM.

\begin{table}[h!]
\centering
\caption{Comparación GFRX+COFB vs AES-128-GCM}
\label{tab:aes_gcm_comparison}
\begin{tabular}{|r|r|r|l|}
\hline
\textbf{Mensaje} & \textbf{GFRX+COFB} & \textbf{AES-GCM} & \textbf{Mejor Esquema} \\
\textbf{bytes} & \textbf{Mbps} & \textbf{Mbps} & \\
\hline
16 & 289.11 & 112.16 & GFRX+COFB 2.58 veces más rápido \\
\hline
64 & 616.90 & 506.48 & GFRX+COFB 1.22 veces más rápido \\
\hline
256 & 889.27 & 1,864.40 & AES-GCM 2.10 veces más rápido \\
\hline
1,024 & 871.33 & 7,116.39 & AES-GCM 8.17 veces más rápido \\
\hline
4,096 & 552.84 & 23,811.68 & AES-GCM 43.1 veces más rápido \\
\hline
16,384 & 220.91 & 54,902.33 & AES-GCM 248 veces más rápido \\
\hline
\end{tabular}
\end{table}

Para mensajes muy pequeños de 16 bytes, GFRX+COFB es 2.58 veces más rápido que AES-GCM debido a menor overhead de inicialización. El punto de cruce ocurre alrededor de 200 bytes. Debajo de esto GFRX+COFB gana, mientras que encima AES-GCM domina gracias a instrucciones AES-NI de hardware.

\subsection{Gráficos de Rendimiento}

La Figura~\ref{fig:throughput_comparison} muestra la comparación visual de throughput.

\begin{figure}[h!]
    \centering
    \includegraphics[width=0.9\textwidth]{../implementacion/throughput_comparison.png}
    \caption{Comparación de throughput entre GFRX+COFB, ASCON y AES-GCM}
    \label{fig:throughput_comparison}
\end{figure}

GFRX+COFB en línea azul lidera en la región de 16-256 bytes. ASCON en línea púrpura mantiene rendimiento estable en todo el rango. AES-GCM en línea naranja tiene inicio lento pero crece exponencialmente con el tamaño.

La Figura~\ref{fig:latency_comparison} presenta latencias para mensajes pequeños en escenario IoT.

\begin{figure}[h!]
    \centering
    \includegraphics[width=0.9\textwidth]{../implementacion/latency_comparison.png}
    \caption{Comparación de latencia para mensajes pequeños de IoT}
    \label{fig:latency_comparison}
\end{figure}

GFRX+COFB muestra las menores latencias en todos los tamaños de mensaje relevantes para IoT de 16-1024 bytes, crítico para aplicaciones en tiempo real.

La Figura~\ref{fig:small_messages} enfoca el rendimiento en el rango óptimo de IoT.

\begin{figure}[h!]
    \centering
    \includegraphics[width=0.9\textwidth]{../implementacion/small_message_performance.png}
    \caption{Rendimiento en mensajes pequeños en escenario IoT típico}
    \label{fig:small_messages}
\end{figure}

En el rango crítico de 16-256 bytes, GFRX+COFB demuestra superioridad clara sobre ASCON y AES-GCM.

\section{Eficiencia de Recursos}

\subsection{Tamaño de Código}

La Tabla~\ref{tab:code_size_comparison} compara el tamaño de implementaciones.

\begin{table}[h!]
\centering
\caption{Comparación de tamaño de código de bibliotecas core}
\label{tab:code_size_comparison}
\begin{tabular}{|l|r|r|r|}
\hline
\textbf{Esquema} & \textbf{Código bytes} & \textbf{Ratio vs} & \textbf{Fit IoT} \\
 & & \textbf{GFRX+COFB} & \textbf{con 32KB flash?} \\
\hline
GFRX+COFB & 6,778 & 1.00× & Sí equivalente a 21\% \\
\hline
ASCON-128 & 3,093 & 0.46× & Sí equivalente a 10\% \\
\hline
GIFT-COFB & 6,703 & 0.99× & Sí equivalente a 21\% \\
\hline
AES-GCM completo & ~25,000 & 3.69× & Ajustado equivalente a 78\% \\
\hline
\end{tabular}
\end{table}

GFRX+COFB es comparable a GIFT-COFB dado que ambos usan COFB. ASCON es más compacto debido a su permutación simple versus Feistel. AES-GCM es 3.7 veces más grande por sus tablas de precálculo y GHASH. Todos los esquemas ligeros caben en 32KB flash.

\subsection{Uso de Memoria RAM}

La Tabla~\ref{tab:ram_usage} presenta el uso de RAM para un mensaje típico de 256 bytes.

\begin{table}[h!]
\centering
\caption{Uso de RAM para mensaje de 256 bytes}
\label{tab:ram_usage}
\begin{tabular}{|l|r|r|}
\hline
\textbf{Componente} & \textbf{GFRX+COFB} & \textbf{ASCON-128} \\
\hline
Contexto persistente & 336 bytes & 320 bytes \\
\hline
Buffers de mensaje & 512 bytes & 512 bytes \\
\hline
Stack locales & ~200 bytes & ~150 bytes \\
\hline
\textbf{Total estimado} & \textbf{~1,048 bytes} & \textbf{~982 bytes} \\
\hline
Porcentaje de 4KB RAM & 25.6\% & 24.0\% \\
\hline
\textbf{Heap allocations} & \textbf{0} & \textbf{0} \\
\hline
\end{tabular}
\end{table}

GFRX+COFB utiliza aproximadamente 1 KB de RAM para mensajes típicos, dejando 3 KB disponibles para lógica de aplicación en dispositivos con 4 KB RAM.

\subsection{Eficiencia de Estado}

La Figura~\ref{fig:efficiency} muestra la eficiencia medida como Mbps por byte de estado.

\begin{figure}[h!]
    \centering
    \includegraphics[width=0.9\textwidth]{../implementacion/efficiency_comparison.png}
    \caption{Eficiencia de estado medida como Mbps por byte de estado}
    \label{fig:efficiency}
\end{figure}

Para mensajes de 256 bytes en caso IoT representativo se obtienen los siguientes valores.

\begin{itemize}
    \item \textbf{GFRX+COFB} alcanza 22.2 Mbps por byte con 320 bits de estado
    \item \textbf{ASCON-128} alcanza 13.3 Mbps por byte con 320 bits de estado
    \item \textbf{AES-128-GCM} alcanza 38.8 Mbps por byte con 384 bits de estado
\end{itemize}

GFRX+COFB ofrece la mejor eficiencia de estado entre esquemas ligeros siendo 67 por ciento mejor que ASCON con el mismo estado de 320 bits.

\subsection{Análisis de CPU}

El profiling mediante gprof reveló la distribución de tiempo de CPU mostrada en la Tabla~\ref{tab:cpu_distribution}.

\begin{table}[h!]
\centering
\caption{Distribución de tiempo de CPU mediante gprof}
\label{tab:cpu_distribution}
\begin{tabular}{|l|r|r|}
\hline
\textbf{Función} & \textbf{Porcentaje Tiempo} & \textbf{Llamadas} \\
\hline
\texttt{gfrx\_encrypt\_block} & 50.0\% & 108,834 \\
\hline
\texttt{cofb\_encrypt} & 50.0\% & 2,561 \\
\hline
\texttt{G\_function} & <0.1\% & 94,628 \\
\hline
\texttt{delta} máscaras & <0.1\% & N/A \\
\hline
\texttt{secure\_compare} & <0.1\% & 1,542 \\
\hline
\texttt{secure\_zero} & <0.1\% & 4,114 \\
\hline
\end{tabular}
\end{table}

La distribución 50/50 entre cifrado de bloque y modo AEAD indica diseño balanceado sin cuello de botella dominante. Las funciones de seguridad tienen overhead despreciable menor a 0.1 por ciento.

\section{Análisis de Seguridad}

\subsection{Efecto Avalancha}

Se midió el efecto avalancha del cifrado GFRX mediante prueba de cambio de un bit.

\begin{itemize}
    \item \textbf{Test} cifrar bloque, cambiar 1 bit del plaintext, cifrar nuevamente, contar bits diferentes en ciphertext
    \item \textbf{Iteraciones} 128 pruebas, una por cada bit del bloque
    \item \textbf{Resultado} 66 de 128 bits cambiados equivalente a 51.56 por ciento
    \item \textbf{Ideal} aproximadamente 50 por ciento según criterio de avalancha estricto
    \item \textbf{Desviación} más 1.56 por ciento dentro de variación estadística normal
\end{itemize}

GFRX cumple el criterio de avalancha, indicando excelente difusión.

\subsection{Verificación de Autenticación}

Se verificó que la autenticación detecta correctamente modificaciones según la Tabla~\ref{tab:authentication_tests}.

\begin{table}[h!]
\centering
\caption{Tests de autenticación}
\label{tab:authentication_tests}
\begin{tabular}{|l|r|r|}
\hline
\textbf{Tipo de Modificación} & \textbf{Tests} & \textbf{Detecciones} \\
\hline
Modificación de tag 1 bit & 10 & 10 equivalente a 100\% \\
\hline
Modificación de ciphertext & 10 & 10 equivalente a 100\% \\
\hline
Modificación de AD & 10 & 10 equivalente a 100\% \\
\hline
Tag completamente incorrecto & 3 & 3 equivalente a 100\% \\
\hline
\textbf{Total} & \textbf{33} & \textbf{33 equivalente a 100\%} \\
\hline
\end{tabular}
\end{table}

El esquema AEAD detecta exitosamente todas las modificaciones maliciosas o accidentales.

\subsection{Resistencia a Ataques}

Basado en análisis publicados, la Tabla~\ref{tab:attack_resistance} muestra la resistencia a ataques criptográficos.

\begin{table}[h!]
\centering
\caption{Resistencia a ataques criptográficos}
\label{tab:attack_resistance}
\begin{tabular}{|l|l|l|}
\hline
\textbf{Tipo de Ataque} & \textbf{Complejidad} & \textbf{Margen de Seguridad} \\
\hline
Criptoanálisis diferencial & $>2^{117}$ & 13 rondas equivalente a 40.6\% \\
\hline
Criptoanálisis lineal & $>2^{102}$ & 19 rondas equivalente a 59.4\% \\
\hline
Birthday attack COFB & $2^{64}$ bloques & Prácticamente inalcanzable \\
\hline
Forgery attack & $2^{-128}$ & Computacionalmente infeasible \\
\hline
Brute force clave & $2^{128}$ & Imposible \\
\hline
\end{tabular}
\end{table}

El nivel de seguridad global es de 128 bits con fortaleza completa.

\section{Discusión de Resultados}

\subsection{Fortalezas Identificadas}

\begin{enumerate}
    \item \textbf{Rendimiento superior en mensajes pequeños} donde GFRX+COFB es 1.5-1.7 veces más rápido que ASCON en el rango crítico de 16-256 bytes que representa el 90 por ciento de tráfico IoT típico.

    \item \textbf{Latencia ultra-baja} de 0.443 µs para 16 bytes permite aplicaciones de tiempo real estrictas en control industrial y sistemas médicos.

    \item \textbf{Uso eficiente de recursos} con solo 6.6 KB de código y 1 KB de RAM, GFRX+COFB cabe en microcontroladores de gama baja como ARM Cortex-M0+ con 32KB flash y 4KB RAM.

    \item \textbf{Sin dependencias externas} lo que proporciona portabilidad máxima a cualquier plataforma con compilador C99.

    \item \textbf{Validación exhaustiva} mediante 1,666 tests automáticos proporcionando alta confianza en correctitud funcional.

    \item \textbf{Seguridad robusta} con márgenes de seguridad mayores a 40 por ciento contra mejores ataques conocidos y 128 bits de seguridad efectiva.
\end{enumerate}

\subsection{Limitaciones Reconocidas}

\begin{enumerate}
    \item \textbf{Rendimiento en mensajes grandes} donde para mensajes mayores a 4KB, ASCON iguala o supera a GFRX+COFB, y AES-GCM con AES-NI es órdenes de magnitud más rápido. Sin embargo estos tamaños de mensaje son raros en IoT.

    \item \textbf{Falta de aceleración hardware} a diferencia de AES-GCM que usa AES-NI. GFRX+COFB depende completamente de implementación software. Implementación futura en FPGA/ASIC podría mejorar significativamente el rendimiento.

    \item \textbf{Esquema relativamente nuevo} dado que GFRX fue publicado en 2023, por lo que ha recibido menos escrutinio criptográfico que ASCON de 2016 o AES de 1998.

    \item \textbf{No estandarizado} a diferencia de AES-GCM con NIST FIPS o ASCON ganador NIST LWC. GFRX+COFB no tiene respaldo de estandarización oficial.

    \item \textbf{Análisis de canal lateral no realizado} dado que no se evaluó resistencia a ataques de timing, potencia o electromagnéticos por estar fuera del alcance de esta tesis.
\end{enumerate}

\subsection{Casos de Uso Recomendados}

La Tabla~\ref{tab:use_cases} presenta los casos de uso óptimos para cada esquema.

\begin{table}[h!]
\centering
\caption{Casos de uso recomendados}
\label{tab:use_cases}
\small
\begin{tabular}{|l|p{10cm}|}
\hline
\textbf{Esquema} & \textbf{Caso de Uso Óptimo} \\
\hline
\textbf{GFRX+COFB} & Sensores IoT, dispositivos portátiles, smart home, dispositivos médicos con mensajes pequeños de 16-256 bytes sin aceleración hardware \\
\hline
ASCON-128 & Aplicaciones ligeras de propósito general con estandarización requerida \\
\hline
AES-128-GCM & Servidores, gateways, dispositivos con AES-NI y mensajes grandes \\
\hline
\end{tabular}
\end{table}

\subsection{Comparación con Objetivos}

La Tabla~\ref{tab:objectives_achievement} evalúa el logro de objetivos específicos.

\begin{table}[h!]
\centering
\caption{Logro de objetivos específicos}
\label{tab:objectives_achievement}
\small
\begin{tabular}{|p{7cm}|p{7cm}|}
\hline
\textbf{Objetivo} & \textbf{Resultado} \\
\hline
Implementar GFRX-128 y COFB en C con estado mínimo 320 bits & Logrado con 561 LOC y 320 bits estado \\
\hline
Validar correctitud mediante tests exhaustivos & Logrado con 1,666 tests y 100\% aprobados \\
\hline
Medir rendimiento en software & Logrado con 289-889 Mbps según tamaño \\
\hline
Obtener estimaciones hardware mediante HLS & No realizado por limitación de alcance \\
\hline
Comparar con AES-GCM, ASCON, GIFT-COFB & Logrado con benchmarks completos \\
\hline
Analizar resistencia a ataques hasta $2^{64}$ consultas & Logrado basado en literatura \\
\hline
\end{tabular}
\end{table}

5 de 6 objetivos específicos fueron logrados completamente. El objetivo de síntesis HLS no se cumplió y se propone como trabajo futuro.

\section{Conclusiones del Capítulo}

Este capítulo presentó la implementación de GFRX+COFB en 561 líneas de código C99 y los resultados experimentales obtenidos, demostrando lo siguiente.

\begin{itemize}
    \item \textbf{Correctitud funcional completa} con 1,666 tests aprobados sin fallos
    \item \textbf{Rendimiento competitivo} siendo 1.5-1.7 veces más rápido que ASCON en mensajes pequeños
    \item \textbf{Eficiencia de recursos} con 6.6 KB código y 1 KB RAM adecuado para IoT
    \item \textbf{Seguridad robusta} con 128 bits efectivos y márgenes mayores a 40 por ciento
    \item \textbf{Calidad de código} con cero fugas memoria y cero warnings
\end{itemize}

GFRX+COFB se posiciona como una alternativa viable y competitiva para dispositivos IoT con recursos limitados, especialmente en aplicaciones dominadas por mensajes pequeños de 16-256 bytes. Las pruebas exhaustivas y análisis comparativo demuestran que el esquema cumple con los requisitos de rendimiento, seguridad y eficiencia necesarios para entornos IoT.
 %Inserta el capítulo 4: Pruebas y Resultados

%%%%%%%%%%%%%%%%%%%%%%%%%%%%%%%%%%%%%%%%%%%%%%%%%%%%%%%%%%%%%%%%%%%%%%

\bibliographystyle{apalike}

\bibliography{Bibliog}
\addcontentsline{toc}{chapter}{Bibliografía}

\end{document}
